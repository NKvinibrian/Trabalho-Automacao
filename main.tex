\documentclass{article}
\usepackage{graphicx} % Required for inserting images

\title{Automação}
\author{% Coloquem os nomes aqui
	Vinicius Maestrelli Wiggers\\[0.5em]
	Vitor Hugo Piontkievitz da Cruz\\[0.5em]
	Nome do Terceiro Autor\\[0.5em]
	Nome do Quarto Autor
}
\date{Setembro 2025}

\begin{document}

\maketitle

\section{Introdução}
Nos recentes séculos, a automação industrial teve uma grande evolução, assim como as várias tecnologias criadas pelo ser humano. A Revolução Industrial é o marco dessa evolução, sendo originada na Inglaterra no século XVIII, embora a automação como um todo é datada de tempos pré-históricos, pois o homem já utilizava e criava mecanismos para sobreviver nestes tempos. Desde então, ela foi aprimorada e aperfeiçoada, chegando aos dias de hoje, em que está presente em muitas grandes indústrias, e ainda sendo usadas por cada vez mais, sendo um elemento de que são altamente dependentes. Na automação, vários dispositivos são utilizados para ajudar no controle dos processos, como os diversos tipos de sensores, que permitem detectar variados elementos e geram sinais de acordo, os relés e contatores, que permitem controlas os sinais de saídas e as correntes, e o Controlador Lógico Programável (CLP) que, utilizando de uma linguagem de programação baseada na lógica booleana, permite fazer o controle de operações sequenciadas e repetitivas na indústria, tendo hardware e software compatíveis.  

\section{Conteúdo}
Aqui conteúdo

\section{Exemplo prático}
Aqui Exemplo prático

\section{Conclusão}
Aqui conclusão


\end{document}
